\chapter{Hardware specifications} \label{APHardware}

An NVIDIA A100 graphics card with the following characteristics shown in table \ref{table:TableHardware} has been used to develop this work.

\vspace{25pt}

\begin{table}[ht]
\centering
\begin{tabular}{|>{\columncolor[HTML]{BFBFBF}}l |l|}
\hline
\textbf{Name} & \cellcolor[HTML]{FFFFFF}Tesla A100-PCIE \\ \hline
\textbf{Year} & 2020 \\ \hline
\textbf{Architecture} & GA100   (Ampere) \\ \hline
\textbf{CUDA capability} & 8.0 \\ \hline
\textbf{CUDA cores} & 6912 \\ \hline
\textbf{Clock MHz} & 1410 \\ \hline
\textbf{Memory GiB} & 39.59 \\ \hline
\textbf{SP peak GFlops} & 19492 \\ \hline
\textbf{DP peak GFlops} & 9746 \\ \hline
\textbf{Peak GB/s} & 1555 \\ \hline
\end{tabular}
\caption{\textbf{Specifications of the NVIDIA A100 GPUs used in this work} \cite{HPCDTU}}
\label{table:TableHardware}
\end{table}

\chapter{Software enviroment} \label{APSoftware}

The software tools used throughout the development of this work are shown in table \ref{APSoftware}.

\vspace{25pt}

\begin{table}[ht]
\centering
\begin{tabular}{|l|l|}
\hline
\rowcolor[HTML]{BFBFBF} 
\textbf{Tool} & \textbf{Version} \\ \hline
\rowcolor[HTML]{FFFFFF} 
Python & 3.8.13 \\ \hline
\rowcolor[HTML]{FFFFFF} 
PyTorch & 2.0.1 \\ \hline
\rowcolor[HTML]{FFFFFF} 
Torchvision & 0.15.2 \\ \hline
\rowcolor[HTML]{FFFFFF} 
Hugging Face Diffusers & 0.16.1 \\ \hline
\rowcolor[HTML]{FFFFFF} 
xFormers & 0.0.19 \\ \hline
\rowcolor[HTML]{FFFFFF} 
Transformers & 4.28.1 \\ \hline
\rowcolor[HTML]{FFFFFF} 
Numpy & 1.24.2 \\ \hline
\rowcolor[HTML]{FFFFFF} 
Scipy & 1.10.0 \\ \hline
\rowcolor[HTML]{FFFFFF} 
Scikit-learn & 1.2.2 \\ \hline
\rowcolor[HTML]{FFFFFF} 
Pandas & 1.5.3 \\ \hline
\rowcolor[HTML]{FFFFFF} 
OpenCV & 4.7.0.72 \\ \hline
\rowcolor[HTML]{FFFFFF} 
Pillow & 9.4.0 \\ \hline
\rowcolor[HTML]{FFFFFF} 
Seaborn & 0.12.2 \\ \hline
\rowcolor[HTML]{FFFFFF} 
Matplotlib & 3.7.1 \\ \hline
\end{tabular}
\caption{\textbf{Versions of the software tools used in the project}}
\label{table:TableHSoftware}
\end{table}



MOSTRAR LAS IMAGES SINTETICAS Y RECORDAR VOLVER AL TEXTO Y RELLENAR LAS XS.