\chapter{Future work and social issues} \label{sec:future}

Overall, the work is solid and successful, offering significant insights into the research questions. However, as with any analysis of this magnitude, there are limitations, areas for improvement and new research questions that arise during the work. Among the main limitations already detected in section \ref{sec:discussion} are the instability in obtaining consistent quality results, the enormous computational requirements and the existence of bias. Therefore, we put forth the following research directions to address these issues and expand the reach of this thesis \footnote{These ideas have not been applied or explored because of the time and resource constraints within which the present work is framed.}.

\textbf{Automatic image selection based on the Fréchet inception distance - FID score}. This metric is used to evaluate the quality and diversity of images generated by text-to-image models. For this purpose, the FID calculates the similarity in the distribution of a set of real images and synthetic images. Consequently, our proposition involves selecting generated images based on their FID scores. In this way, even if the model generates images of low quality or with defects, it would be possible to eliminate them or select only those with higher quality. However, it is important to note that accomplishing this task is not straightforward and necessitates longer execution times and establishing an appropriate threshold through experimental means.

\textbf{Improving and optimising subject-driven methods}. Both Dreambooth and Textual inversion authors propose improvements to their methods. These could improve both execution times and the representations obtained in custom text-to-image models. In particular, the researchers responsible for Dreambooth highlight the following flaws \cite{ruiz2023dreambooth}: errors in the synthesis of the environment in which the subject is located, changes in the subject's appearance depending on the environment and overfitting to the input images. All these flaws have been observed in the present work \ref{sec:discussion}. On the other hand, Textual inversion warns about anatomical faults \cite{gal2022image}, which we have also detected. Finally, our Stable Diffusion prompt approach can also be improved through an automatic selection method for higher-quality prompts.

\textbf{Use of better and more advanced text-to-image models}. With them, better-quality images could be obtained. One possible approach is replicating the experiments using an upgraded version of Stable Diffusion, such as \textit{v2-1} or other models, such as Imagen \cite{saharia2022photorealistic}. The reason for using version \textit{v1-5} in the present work is its wide compatibility with the libraries used.

\textbf{Optimisation of execution times and reduction of environmental impact}. We propose to address this aspect as we recognise it as a significant limiting factor in conducting comprehensive and rigorous experimentation. Specifically, the primary dataset analysed, Oxford-IIIT Pet, only has 37 classes, so it has been feasible to experiment on it. However, in more intricate datasets encompassing hundreds of categories, the execution times escalate substantially. For example, applying Dreambooth to 100 classes with the hardware and environment used in this work would take more than 75 hours (approximately 45 minutes per class). Moreover, generating images and training the associated task would entail additional tens of hours of computation. Consequently, we deem it crucial to address this aspect due to the considerable environmental impact associated with the energy-intensive hardware requirements.

\textbf{Bias reduction}. It is a pressing concern in text-to-image models, as expounded upon in section \ref{sec:discussion}. These models, particularly large-scale ones, often inherit biases and stereotypes from the training data they rely upon. We propose that further research be conducted to address this issue comprehensively. In particular, it is crucial to investigate the impact of subject-driven augmentation techniques, as these approaches can potentially transfer biases from text-to-image models to the associated models. Given the social implications involved, it is essential to explore strategies to reduce such biases and mitigate potential societal challenges that may arise as a result.